\documentclass[a4paper, 12pt]{article}

% -----------------------------------------------------------------------------
%
%       Ataskaitos šablonas
%
%       Redagavo: Domantas Petrauskas domantas.pet@gmail.com
%       Autorius: Tomas Rekašius      tomas.rekasius@vgtu.lt
%
%       Sukurta: 2013-12-27
%       Versija: 1.0.0
%
% -----------------------------------------------------------------------------


% -----------------------------------------------------------------------------
%  PREAMBULĖ
% -----------------------------------------------------------------------------

\usepackage[utf8]{inputenc} % naudojama, kai .tex failas UTF-8 koduotės
\usepackage[L7x]{fontenc} % nurodoma lietuviško teksto koduotė Latin-7
\usepackage[lithuanian]{babel}  % nurodoma, kad dokumentas yra lietuviškas

\usepackage{lmodern}  % dokumente naudojamas šriftas Latin Modern
\usepackage{microtype}  % optimizuojami atstumai tarp raidžių žodyje
\usepackage{indentfirst}  % atitraukiama pirmoji naujo skyriaus eilutė
\usepackage{icomma} % po kablelio skaičiaus viduryje nebus tarpo
\usepackage{amsmath, amssymb, amsthm} % matematiniai simboliai ir konstrukcijos
\usepackage{graphicx} % grafinių failų įterpimas ir kiti nustatymai
\usepackage{svg}  % vektorinės grafikos failų palaikymas
\usepackage{xcolor} % naudojamas teksto spalvoms reguliuoti
\usepackage{booktabs} % reikalingas tvarkingoms lentelėms sudaryti
\usepackage{multirow} % paketas kelių eilučių apjungimui lentelėse
\usepackage{array}  % paketas papildomiems lentelių nustatymams
\usepackage{listings} % programinio kodo įterpimas ir formatavimas
\usepackage{lipsum} % Lorem Ipsum teksto generatorius
\usepackage[numbered]{matlab-prettifier}  % matlab kodo stilius
\usepackage{caption}  % paveiksliukų ir lentelių užrašų formavimas
\usepackage{geometry} % paraščių ir kitų lapo parametrų nustatymai
\usepackage{hyperref} % interaktyvioms nuorodoms dokumente sukurti
\usepackage{multicol} % paketas stulpeliams formuoti

\geometry{  % paketo geometry parametrų nustatymai
    left = 2.5cm,
    right = 1.5cm,
    top = 2.0cm,
    bottom = 2.0cm
}

\captionsetup{  % paketo caption parametrų nustatymai
    format = hang,
    labelfont = bf,
    font = small,
    tablename = lentelė,
    figurename = pav,
    labelsep = period
}

\hypersetup{  % paketo hyperref parametrų nustatymai
    unicode = true,
    linktocpage = false,
    colorlinks = true,
    linkcolor = red, % spausdinant pakeisti į black
    citecolor = blue
}

\lstset{  % paketo listings/matlab-prettifier parametrų nustatymai
  style              = Matlab-editor, % pakeisti į Matlab-bw spausdinant nespalvotai
  basicstyle         = \mlttfamily,
  escapechar         = ",
  mlshowsectionrules = true,
  firstnumber=last
}

% ------------- PAPILDOMI PAKETAI IR NUSTATYMAI -------------

\lstMakeShortInline[style=Matlab-editor]"   % simbolis inline matlab kodo įterpimui
\linespread{1.3}    % nustatomas 1,5 tarpas tarp eilučių


% -----------------------------------------------------------------------------
%  DOKUMENTO PRADŽIA
% -----------------------------------------------------------------------------

\begin{document}

% ------------- VIRŠELIS -------------

\begin{titlepage}

    \centering

    \begin{figure}
        % \includegraphics[height=2cm]{vgtu_logo_lt}  % Lietuviškas VGTU logotipas
        \includesvg[height=2cm]{vgtu_logo_en} % Angliškas VGTU logotipas
        \centering
    \end{figure}

    \textsc{\Large Vilniaus Gedimino Technikos Universitetas}\\[.5\baselineskip] % Universiteto pavadinimas
    \textsc{\large X fakultetas}\\[.5\baselineskip] % Fakulteto pavadinimas
    \textsc{\large Y katedra}\\[.5\baselineskip] % Katedros pavadinimas

    \vspace{\fill}

    {\Large\bfseries Y darbas Nr. X}\\[1.0\baselineskip] % Darbo pavadinimas
    {\Large Ataskaita}

    \vspace{\fill}

    \begin{flushright}
        \begin{tabular}{rl}
            Atliko:       & stud. Vardas Pavardė\\
            Tikrino:      & prof. Vardas Pavardė\\[1cm]
            Įvertinimas: & \dotfill{}\\
            & \multicolumn{1}{c}{\large\textsuperscript{(pažymys, dėstytojo parašas)}}
        \end{tabular}
    \end{flushright}

    \vspace{\fill}

    Vilnius, \the\year{}

\end{titlepage}

% ------------- TURINYS -------------

\newpage

\tableofcontents

\newpage

% ------------- UŽDUOTIS -------------

\section{Užduoties analizė}
\subsection{Darbo tikslas}
\lipsum[1]

\subsection{Darbo priemonės}
\lipsum[2]

\subsection{Užduotis}
\lipsum[3]

% ------------- DARBO EIGA -------------

\section{Darbo eiga}
\lipsum[4]

% ------------- REZULTATAI -------------

\section{Rezultatai}
\lipsum[5]

% ------------- IŠVADOS -------------

\section{Išvados}
\lipsum[6]

% ------------- LITERATŪROS SĄRAŠAS -------------

\newpage

\bibliography{}

\phantomsection{}
\addcontentsline{toc}{section}{Literatūros sąrašas}  % Bibliografija į turinį automatiškai neįtraukiama.

\begin{thebibliography}{99}
\bibitem{pavarde} Vardas Pavarde. Pavadinimas. Miestas: Leidykla, Metai. Puslapių skaičius., ISBN xxxx-xx-xxx-x.
\end{thebibliography}

\end{document}
